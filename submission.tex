\documentclass{article}
\usepackage{amsmath}
\usepackage{esint}
\usepackage{svg}
\usepackage{nicefrac}
\svgpath{{svg/}}
\title{\textbf{ECE Lab 3A: FFT}}
\date{\textit{2019-11-25}}
\author{\textit{Varun Iyer} \hspace{2em}$\cdot$ \hspace{2em} \textit{Rajan Saini}}
\begin{document}
\maketitle
\section{Purpose}
	Our primary goal in this lab was to determine the frequency of an audio signal in under 50 milliseconds.
	There are two necessary tasks to complete this goal.
	First, we must incorporate the Nexsys board’s on-board microphone into our hardware design.
	Second, we must optimize the Fast Fourier Transform algorithm provided to us to operate within this time limit.
\section{Methodology}
	\subsection{Microphone Incorporation}	
		
	\subsection{FFT Optimization}
		\subsubsection{Twiddle Coefficients}
			Initially, we noticed that the slowest aspect of the program was the calculation of the twiddle coefficients, which involve multiple $\sin$ and $\cos$ calculations and complex multiplication.
			The first step we took was to save the results of $\sin$ and $\cos$ computation for the same values of \texttt{k} and \texttt{b} so we didn’t have to compute them repeatedly.
			For further improvements, we decided to avoid using high-order and computationally expensive Taylor expansions for trigonometric computations.
			We replaced these with a lookup table and used trigonometric identities and low-order, small-angle Taylor approximations for interpolation.
			We saw substantial improvements from this (we saw 75\% of the computation time melt away), but realized that we could make even more time gains by minimizing the amount of floating point operations and needless $\pi$ multiplications throughout this calculation.
			Instead of passing in an angle as a floating point value, we entered it as a ratio between two integers, \texttt{k} and \texttt{b}.
			We could now perform all $2\pi$ and $\nicefrac{\pi}{2}$ modulo using integer ratios and save floating point operations for the final multiplications.
		\subsubsection{Division Optimization}
			We also saw that floating-point divisions were very computationally expensive, despite using an FPU. 

			We looked for instances of float division by a number of the form $2^n$ and subtracted the exponential bits by $n$ instead. This resulted in a 10 ms speedup. 
			Whenever this did not hold, we did our division using fixed-point arithmetic so that we could divide by a power of 2 when converting back to a float.
		\subsubsection{Imaginary Removal}
			We noticed that our sampling is entirely in the real numbers and imaginary numbers are passed in as zeroes.
			For this reason, we removed all reordering and unnecessary computation of the imaginary component of the sample until the calculation of $N=2$ DFTs, when twiddle multiplication results in non-zero imaginary components.
\end{document}
